\documentclass[12pt,a4paper]{article}
\usepackage[utf8]{inputenc}
\usepackage[russian]{babel}
\usepackage[left=2.00cm, right=2.00cm, top=2.00cm, bottom=2.00cm]{geometry}
\linespread{1.25}
\usepackage{setspace}
\usepackage{indentfirst}
\setlength{\parindent}{1.25cm}
\let\paragraph\ignorespaces
\usepackage{tabularx}
\usepackage{multirow}
\usepackage{graphicx}



\begin{document}
	
\begin{titlepage}
	
\begin{center}
	\large Университет ИТМО\\[5cm]
	\LARGE Практическая работа №1\\
	\normalsize по дисциплине <<Визуализация и моделирование>>\\[5cm]
\end{center}
\begin{flushright}
		\begin{minipage}{0.6\textwidth}
		\begin{flushleft}
			\large
			\singlespacing 
			\textbf{Автор:} Редичкина Анна Максимовна\\
			\textbf{Поток:} ВИМ 1.1\\
			\textbf{Группа:} К3242\\
			\textbf{Факультет:} ИКТ\\
			\textbf{Преподаватель:} Чернышева А.В.
		\end{flushleft}
	\end{minipage}
\end{flushright}

\vfill

\begin{center}
	{\large Санкт-Петербург, \the\year{ г.}}
\end{center}
 
\end{titlepage}
\normalsize


\section{Описание датасета}
'Graduate Admission 2' содержит в себе иноформацию об академической успеваемости школьников, их учебных достижениях. На основе этих параметров был расчитан шанс поступления в один из пяти университетов, куда хочет попасть абитуриент. В датасете 8 колонок и 500 строк. 

\section{Содержание колонок}
\begin{table}[h]
\caption{Columns}

\begin{tabular}{|p{4cm}|p{8cm}|p{4cm}|}
	\hline Название колонки & Описание данных & Тип данных \\ \hline
	Serial No. & Порядковый номер абитуриента & целочисленный тип данных \\ \hline
	GRE Score & Результаты экзамена GREE (тест, который необходимо сдавать для поступления в аспирантуру, магистратуру или иной последипломный курс в вузе США и ряда других стран). Максимальный балл - 340. & целочичсленный тип данных \\ \hline
	TOEFL Score & Результаты экзамена на знание английского языка (максимальный балл - 120). & целочисленный тип данных \\ \hline
	University rating & Позиция университета, куда подает заявление абитуриент, в рейтинге (топ-5 университетов). & целочисленный тип данных \\ \hline
	SOP & Statement of Purpose Strength (оценка за рекомендательное письмо и заявление о намерениях). Максимальный балл - 5. & числа с плавающей точкой\\ \hline
	LOR & Letter of Recommendation (оценка за заявление о намерениях). Максимальный балл - 5. & числа с плавающей точкой\\ \hline
	CGPA & Средний балл атестата школьника. Максимальный балл - 10. & числа с плавающей точкой \\ \hline
	Research & Данные показывают наличие исследовательского опыта (1-есть, 0 - нет) & целочисленный тип данных \\ \hline
	Chance of Admit & Шанс попадания в университет (от 0 до 1) & числа с плавающей точкой \\ \hline


\end{tabular} 
\end{table}

\section{Глобальные задачи, которые можно решить на данном датасете}
1. Обучить нейросеть определять вероятность поступления в вуз на основе академической успеваемости.  \\

2. Рассмотреть влияние каждого параметра на вероятность поступления и выявить наиболее и наименее важные из них. \\

3. Узнать какие параметры наиболее важны для каждого из пяти университетов. К примеру, для каких-то вузов может быть более значим опыт в исследовательской деятельности чем средний балл аттестата, и исследование датасета поможет это выявить.

   













\end{document}